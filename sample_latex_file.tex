%
%  PREAMBLE 
%
%

%
% Note that lines preceded by "%" won't get formatted. So I can put comments there.
%

\documentclass[11pt]{amsart} 
% 11pt is text size. You may want 12pt instead, but not smaller that 11pt please.  
% amsart gives access to many latex math symbols (from the AMS = American Mathematical Society)
\pagestyle{empty}  % Avoids page numbers

% Some self-defined commands, as shorthand for the standard latex commands. 
% E.g. I just type \N rather than having to type \mathbb{N} all the time. 

\newcommand{\N}{\mathbb{N}}  % Natural numbers
\newcommand{\Z}{\mathbb{Z}}  % Integers
\newcommand{\R}{\mathbb{R}}  % Real numbers
\newcommand{\C}{\mathbb{C}}  % Complex numbers
\newcommand{\p}{\partial}  %  for partial derivatives
\newcommand{\f}{\frac} % fractions

% The stuff below is for augmented matrices. I got this off the web. I have no idea what it is doing, as long as it works. 
%
\makeatletter
\renewcommand*\env@matrix[1][*\c@MaxMatrixCols c]{%
  \hskip -\arraycolsep
  \let\@ifnextchar\new@ifnextchar
  \array{#1}}
\makeatother




% --------------
% THE DOCUMENT
% --------------

\begin{document}
\thispagestyle{empty}   

\begin{center}
{\bf \LARGE Sample Latex File}\\
% \\ gives a line break
% \bf means bold face
% \LARGE, \Large, \large, \small changing the  
{\bf \large \date{today}   }
\end{center}
% \today produces today's date. You can, of course, also put a fixed date
\vskip.1in
% This creates extra vertical space of 0.1 inches
%
Your TA will love it if you hand in a typed version of your assignment! The text below 
gives some ideas of how to use the latex symbols. Lots of additional examples, and list of commands, can be found on the internet. \\

To turn the raw Latex file (the \emph{source file}), with ending ".tex", into a pdf (or dvi) file, you'll have to format it. Eventually, you probably want to have some latex installation on your computer (so you don't need 
access to the web).  You can download everything for free from the internet, but it may take some time to get it running.\footnote{ Don't ask me, I'm not good with computers. I managed to do it myself, but forgot.} \\

Alternatively, its also possible to use online compilers. For the homework problems, this may be quite enough. For example, if you use 
\begin{quote}
{\tt \small http://latex.informatik.uni-halle.de/latex-online/latex.php}
\end{quote}
(switch to English),  you just copy-and-paste the entire raw text (=source file) into the window, press \emph{compile}, and the scroll down to retrieve the pdf file. Note that you sometimes have to compile twice, to get the cross-references right.  For example, you could try it with the latex source file for this document, titled \verb!sample_latex_file.tex!.

\section{Latex basics}
There's lots of information on the internet, I don't have to repeat it. Just enter `latex for beginners' or similar into your search engine. 

Important: any command "\verb!\begin{...}!" gets  closed with "\verb!\end{...}!",
$\verb!\left{..}!$ gets closed with $\verb!\right{..}!$ , and any curly bracket that opens up  "\verb!{!" also gets closed, "\verb!}!". If there are too many errors of that sort, the compiler will balk, or the outcome is a mess. 

Also, whatever comes after \verb!%! on a given line in the source file, the compiler will ignore. We can e.g. use this to put comments into the source file,  as reminders for ourselves.  
 

\section{Matrices and vectors}
Here is a typical matrix. 
$$ \left( 
\begin{array}{ccc}   % Here ccc means that we have three columns, all centered. 
7 & -1 & 3 \\ -10 & 5 & 2
\end{array}\right)
$$
We can also do this :
$$ \left( 
\begin{array}{cccc}   % Here cccc means that we have four columns, all centered. 
a_{11} & a_{12} & \cdots & a_{1n} \\ 
a_{21} & a_{22} & \cdots & a_{2n} \\
\vdots & \vdots  &            & \vdots \\
a_{m1} & a_{m2} & \cdots & a_{mn}
\end{array}\right)
$$
%
A column vector is a matrix with one column, as in the following equation
%
$$\left( 
\begin{array}{ccc}   
7 & -1 & 3 \\ -10 & 5 & 2
\end{array}\right)
 \left( 
\begin{array}{c}   
x \\ y \\ z 
\end{array}\right)
=
 \left( 
\begin{array}{c}   
17 \\ -1 
\end{array}
\right)
$$
Here is the same equation written as a system of equations. 
$$ 7 x -y +3 x    = 17,$$
$$ -10 x +5y+2z  = -1.$$
That doesn't look so great, does it? Better is: 
\begin{align*}
7 x -y +3 x   & = 17\\
-10 x +5y+2z & = -1.
\end{align*}
Here I used the \verb!\begin{align*}!, 
\verb!\end{align*}! environment in order to align the equations nicely, along 
the = signs. (Without the stars, one gets numbered equations -- see below.) 

At some stage, you may also need `augmented matrices', such as 
\[
\begin{pmatrix}[ccc|cc]
A_{11}& \cdots & A_{1n}& b_1&  c_1 \\ 
\vdots & & \vdots   & \!\!\vdots &  \vdots \\ 
A_{m1}& \cdots & A_{mn}& b_m &  c_m
 \end{pmatrix}
\]
For this to work, you must have a certain blurb in the preamble -- take a look at the preamble to the source file. 
Ask me not what this is doing, I just copied it off the web. (Whenever you don't know how to typest something, you'll usually find the answer on the internet.)





A few more things: 
\begin{itemize}
\item 
%
$$ \underbrace{1+1+\ldots +1}_{n\ \mbox {\tiny times} }$$
%
Here I used \verb!\mbox! to write text in the equation. 
\item 
Sets: 
$$ A\subseteq B,\ \ \ A\times B,\ \ x\in A,\ \ \ \bigcap_{i=1}^n A_n=\emptyset,\ \ \ \ \ 
\N=\{1,2,3,\ldots\}.$$
\item 
Sums, fractions: 
$$ 6\cdot \sum_{i=1}^n \f{1}{n^2}=\pi^2.$$
\item
Other fonts: $\mathfrak{A},\mathfrak{B}$ or $\mathcal{A},\mathcal{B}$ or 
$\mathsf{A},\mathsf{B}$, also in lower case: $\mathfrak{a}, \mathfrak{b}$, etc.. 
Greek letters: $\alpha,\beta,\gamma,\delta$, and so on. 
\end{itemize}


\section{Proofs}
%
Here are some symbols that you may encounter in analysis: 
%
\[ \forall \epsilon >0 \exists \delta >0 \forall x\colon |x-a|<\epsilon \Rightarrow 
|f(x)-f(a)|<\delta.\]
%
This looks  jumbled because of bad spacings. Using "\verb!\!" and "\verb!\,!", we can adjust the spacings: 
%
\[ \forall \epsilon >0\  \exists \delta >0\  \forall x\colon |x-a|<\epsilon\,  
\Rightarrow\,  
|f(x)-f(a)|<\delta.\]
That's better! 
\newpage % I'm starting a new page! 

Here is a typical proof. 
\vskip.2in

{\bf Theorem.}
In the field $\C$ of complex numbers, $1=-1$. 
\vskip.1in
\begin{proof}
We use $i=\sqrt{-1}$, as follows: 
\begin{align*} -1=i\cdot i & \Rightarrow  -1=\sqrt{-1}\cdot \sqrt{-1}\\
& \Rightarrow  -1=\sqrt{(-1)\cdot (-1)}\\
& \Rightarrow -1=\sqrt{1}\\
&\Leftrightarrow -1=1. 
\end{align*}
\end{proof}
\noindent{\bf Remark:} This proof (from lecture notes of Dror Bar-Natan) is completely wrong, of course!! But at least, it is neatly arranged. Could also write it as a one-liner: 
$$ -1=i\cdot i=\sqrt{-1}\cdot \sqrt{-1}=\sqrt{(-1)\cdot (-1)}=\sqrt{1}=1.$$


\section{Numberings}
If there are several parts to an argument, you can use the \verb!\itemize! or 
\verb!\enumerate! environments to organize them. \verb!\itemize! gives
\begin{itemize}
\item First item, 
\item Second item. 
\end{itemize}
while \verb!\enumerate! produces
\begin{enumerate}
\item First item,
\item Second item.
\end{enumerate}
%
If you want to enumerate displayed equations, instead of using \verb!$$!, use the \verb!{equation}! environment. (Latex will put the numbers automatically, but you can also customize it.) Use \verb!\label! to give your equation a label, and \verb!\ref! to refer to it later. (You have to compile twice so that latex will understand.) For instance, 
 %
\begin{equation}\label{continuity}
\forall \epsilon >0\  \exists \delta >0\  \forall x\colon |x-a|<\epsilon\,  
\Rightarrow\,  
|f(x)-f(a)|<\delta.
\end{equation}
or 
\begin{equation}\label{einstein}
E=m c^2
\end{equation}
will be referred to as (\ref{continuity}) and \eqref{einstein}. 
% \eqref is a shortcut

\vskip.6in
\begin{center}
\Large 
\it
Enjoy!
\end{center}
\end{document}






